% -*- mode: latex; mode: flyspell; mode: auto-fill -*-

\documentclass[12pt,onecolumn]{memoir}
\usepackage[english,activeacute]{babel}
\usepackage[utf8]{inputenc}
\usepackage[hyperindex,backref,bookmarks,linktocpage]{hyperref}

\usepackage[T1]{fontenc}

\usepackage{textcomp}
\usepackage[sc]{mathpazo}
\usepackage{helvet}
\linespread{1.05} % Palatino needs more leading (space between lines)

\usepackage{enumerate}
\usepackage{epsfig}
\usepackage{subfig}
\usepackage{amsmath}
\usepackage{multirow}
\usepackage{color}
\usepackage{setspace}
\usepackage{rotating}
\usepackage{colortbl}

\hypersetup{ 
  pdfauthor = {Alberto Ros Bardisa}, 
  pdftitle = {Apuntes de Ampliación de Estructuras de Computadores}, 
  pdfkeywords = {arquitectura de computadores, rendimiento, segmentación, memoria}, 
}

\DeclareTextFontCommand{\c}{\sffamily \slshape}

\setstocksize{297mm}{210mm}
\settrimmedsize{297mm}{210mm}{*}
\setlength{\trimtop}{0pt}
\setlength{\trimedge}{\stockwidth}
\addtolength{\trimedge}{-\paperwidth}
\setlrmarginsandblock{35mm}{25mm}{*}
\setulmarginsandblock{45mm}{50mm}{*}
\setheadfoot{\onelineskip}{3\onelineskip}
\setheaderspaces{*}{1.2\onelineskip}{*}
\checkandfixthelayout

%\parskip 	10pt

\pagestyle{ruled}

\setsecnumdepth{all}

\renewcommand{\topfraction}{1}
\renewcommand{\bottomfraction}{1}
\renewcommand{\textfraction}{0}
\renewcommand{\floatpagefraction}{0}

\makechapterstyle{tesis}{
  \setlength{\beforechapskip}{70pt}
  \renewcommand*{\chapnamefont}{%
    \normalfont\Huge\scshape\raggedleft}
  \renewcommand*{\chaptitlefont}{%
    \normalfont\Huge\bfseries\sffamily\raggedleft}
  \renewcommand*{\afterchapternum}{%
    \par\hspace{1.5cm}\hrule\vskip\midchapskip}
}

\chapterstyle{tesis}

\setsecheadstyle{\Large\bfseries\sffamily\raggedright}
\setsubsecheadstyle{\large\bfseries\sffamily\raggedright}
\setsubsubsecheadstyle{\normalsize\bfseries\sffamily\raggedright}
\setparaheadstyle{\normalsize\bfseries\sffamily}
\setsubparaheadstyle{\normalsize\bfseries\sffamily}

\begin{document}

\hyphenation{ge-ne-ra-dos}

\sloppy
\widowpenalty=10000
\clubpenalty=10000

\title{Apuntes de Ampliación de Estructuras de Computadores}

\author{Alberto Ros Bardisa}

% Begin of title page
\thispagestyle{empty}

\begin{center}

\begin{figure}
\centering
%  \epsfig{file=eps/escudo_universidad,width=6cm}
%  \epsfig{file=eps/escudo.ps,width=6cm,viewport=0 -40 690 639}

  \Large

  %\vspace{1cm}

  \textsc{Universidad de Murcia}

  \large

  Departamento de Ingeniería y \\
  Tecnología de Computadores

  \vspace{3cm}

\end{figure}

\huge

\textbf{Apuntes de\\ Ampliación de Estructura de Computadores}

\vspace{7cm}

\normalsize

Alberto Ros Bardisa

\vspace{0.5cm}

Murcia, septiembre de 2020

\end{center}
% End of title page

\cleardoublepage 

\chapter*{Resumen}
\addcontentsline{toc}{chapter}{Resumen}

Estos apuntes pretenden servir como guía para la asignatura
``Ampliación de Estructura de Computadores'' del Grado en Ingeniería
Informática y Programación Conjunta de Estudios Oficiales Grado
en Matemáticas y Grado en Ingeniería Informática.


\setcounter{chapter}{-1}

\def\figurename{Figura}
\def\tablename{Tabla}
\def\chaptername{Tema}

\chapter{Presentación de la asignatura}
\label{cap:presentacion}

Esta asignatura pertenece al área de ``Arquitectura y Tecnología de
Computadores'' y es la tercera de una serie de asignaturas del grado
destinadas a comprender cómo funcionan los computadores y cómo
hacerlos más eficientes.

En ``Fundamentos de Computadores'' se estudia la codificación de la
información y los sistemas digitales combinacionales. En ``Estructura
y Tecnología de Computadores'' se estudian sistemas digitales
secuenciales, un procesador básico y la jerarquía de memoria. En
``Ampliación de Estructura de Computadores'' se hace énfasis en el
rendimiento del computador, se estudia un procesador más avanzado
(segmentado) y aspectos avanzados de la jerarquía de
memoria. Finalmente, en ``Arquitectura y Organización de
Computadores'' se estudian aspectos como el diseño de un procesador
superescalar, la sincronización, el protocolo de coherencia de cachés
y los modelos de consistencia de memoria.

En particular, en ``Ampliación de Estructura de Computadores'' vamos a
ver cómo se pueden mejorar las prestaciones de los computadores por
primera vez en el grado. Hasta el momento nos hemos centrado en tener
un computador que funcione correctamente, sin mirar mucho su
rendimiento. En este curso vamos a construir un ordenador mas potente,
gracias a mejoras ``arquitectónicas''. Se puede decir, por tanto, que
en esta asignatura comienza lo que se conoce por \emph{arquitectura de
  computadores}. No solo vamos a aprender a identificar qué diseño es
mejor que otro de forma ``cualitativa'' sino que también vamos a
aprender a decir de forma ``cuantitativa'' cuantas veces mejor es un
diseño que otro.

Nota: La asignatura tiene asignados 6 ECTS. Es decir, se requieren
unas 150 horas de trabajo por parte del alumno: 60 horas presenciales
y 90 horas no presenciales. Esto significa que por cada hora
presencial, el alumno debe dedicar 1,5 horas más al estudio de la
asignatura.

\section{Introducción}
\label{sec:introduccion_presentacion}

Ley de Moore: Los continuos avances en la escala de integración
permiten reducir el tamaño de los transistores y por tanto el número
de transistores que hay dentro de un chip. El número de transistores
por chip se dobla cada año y medio o dos años.

Las mejoras en el rendimiento de los computadores desde su inicio han
sido debidas tradicionalmente a dos motivos principales: mejoras
debidas a la tecnología y mejoras debidas a la arquitectura. Hasta los
70, las mejoras de ambos eran similares. A partir de los 70 las
mejoras en la tecnología han dominado.



%%%%%%%%%%%%%%%%%%%%%%%%%%%%%%%%%%%%%%%%%%%%%%%%%%%%%%%%%%%%%%%%%%%%%%%%%%%%%%%%%%%%%%%%%%%%

\chapter{Análisis de prestaciones en arquitectura de computadores}
\label{cap:prestaciones}



\section{Introducción}
\label{sec:introduccion_prestaciones}

Ejemplo de rendimiento de un coche ¿Velocidad? El equivalente en el
computador serían los MIPS, MFLOPS.. buena medida? No. Ejemplo rutas diferentes. Lo que importa es
el tiempo dado un origen y un destino.

\section{Definición de rendimiento y métricas populares}

\section{Tiempo de ejecución}

\section{La Ley de Amdahl}

\section{Cómo comparar resultados}

\section{Programas de prueba (benchmarks)}

%%%%%%%%%%%%%%%%%%%%%%%%%%%%%%%%%%%%%%%%%%%%%%%%%%%%%%%%%%%%%%%%%%%%%%%%%%%%%%%%%%%%%%%%%%%%

\chapter{Segmentación básica}
\label{cap:segmentado}


\section{Introducción}
\label{sec:introduccion_segmentado}

Concepto de segmentación. Segmentación para la ejecución de
instrucciones: camino de datos y control segmentado. Riesgos de la
segmentación (estructurales, de datos y de control). Excepciones y su
implementación.

\subsection{El juego de instrucciones DLX}

Repaso de instrucciones. Remarcar diferencias con MIPS (ETC): blt -> slt \& beqz

\subsection{Procesador multiciclo para DLX}

Repaso de etapas multiciclo (ETC), pero adaptado a DLX. Ejemplo de
un load, que es el más largo (5 ciclos). El resto de instrucciones se adaptan a éste. 

\section{Segmentación para la ejecución de instrucciones}

Describir segmentación a grandes rasgos, o de forma esquemática. Los
detalles de implementación se verán al final.

Ejercicio cuantitativo de segmentación.

\section{Problemas de la segmentación: los riesgos}

Si tuvieramos una fábrica de coches la segmentación ya funcionaría
bien y tendríamos un cauce ideal, produciendo un coche por tiempo de
cada etapa.

Pero las instrucciones son más complicadas. No todas son iguales. Y la
ejecución de una puede depender de la otra.

Hay 3 tipos de riesgos: estructurales, de datos y de control. 

\subsection{Riesgos estructurales}

Opciones: duplicar, parar, evitar sw (separar o nops).

\subsection{Riesgos de datos}

Opciones: parar, forwarding, evitar sw (separar o nops).

\subsection{Riesgos de control}

Opciones: parar, adelantar calculo salto, predecir, evitar sw (delay-slot o nops).

Detalle de la implementación. ¿Cómo parar ante riesgos (y adelantar el
calculo del salto)?

Ejercicios de delay en calculo de tiempo.

%%%%%%%%%%%%%%%%%%%%%%%%%%%%%%%%%%%%%%%%%%%%%%%%%%%%%%%%%%%%%%%%%%%%%%%%%%%%%%%%%%%%%%%%%%%%

\chapter{Segmentación avanzada y predicción de saltos}
\label{cap:avanzada}


\section{Introducción}
\label{sec:introduccion_avanzada}

Mitigación de los riesgos de datos: adelantamientos. Mitigación de los riesgos de control (predicción de saltos estática y predicción de saltos dinámica). Segmentación DLX de punto flotante. Cauces con terminación fuera de orden.

\section{Predicción de saltos}

Observación 1: los saltos tienden a tener el mismo comportamiento.

Observación 2: saltos correlacionados.

\subsection{Estática}

Algo más sobre predicción estática saltos para alante o para atrás o comparación == 0 o != 0.

\subsection{Contador saturado por salto}

Ejemplo: bucle.

Aliasing.

\subsection{Correlación}

ejemplo: if d == 0
ejemplo: found = find(elem); if (found) ...  if a[i] = elem.

historia global.

gshare.

%%%%%%%%%%%%%%%%%%%%%%%%%%%%%%%%%%%%%%%%%%%%%%%%%%%%%%%%%%%%%%%%%%%%%%%%%%%%%%%%%%%%%%%%%%%%

\chapter{Sistema de memoria de altas prestaciones}
\label{cap:memoria}


\section{Introducción}
\label{sec:introduccion_memoria}

Introducción a los sistemas de memoria. Diseño de una jerarquía de caches de alto rendimiento: reducción de la tasa de fallos,  reducción de la penalización por fallo y reducción del tiempo de acierto en caches. Organizaciones de la memoria principal. Memoria virtual. Técnicas de traducción rápida de direcciones virtuales. Protección de memoria.

\section{Evaluación del Rendimiento de la Jerarquía de Memoria}

\section{Reducción de la Tasa de Fallos de Caché}

\subsection{Clasificación de fallos de caché}

Podemos distinguir los siguientes motivos por los que se 
producen fallos en caché:

Forzosos: el primer acceso a un bloque de memoria no puede 
estar en la caché (poco efecto en programas grandes)
- Llamados fallos de arranque en frío o de primera referencia

Capacidad: la memoria caché no tiene el tamaño suficiente para 
contener todos los bloques necesarios
- En un momento determinado uno de los bloques que se han 
utilizado tiene que dejar hueco a otro (la caché está llena)
- Se produce fallo si se vuelve a referenciar el bloque desalojado

Conflicto: la memoria caché no es totalmente asociativa 
- Aciertos en una caché totalmente asociativa que se vuelven 
fallos en una asociativa por conjuntos de 
n-vías se deben a más de n peticiones sobre algunos conjuntos

\subsection{Optimizaciones hardware}

\subsubsection{Aumento del tamaño de bloque}

Y del tamaño de la caché. De hecho la primera figura (pag 29) lo contempla.

\subsubsection{Aumento de la asociatividad}

\subsubsection{Caché de víctimas}

\subsubsection{Búsqueda anticipada de datos e instrucciones (prefetching)}

\subsection{Optimizaciones del compilador}

\subsubsection{Combinación de arrays (merging arrays)}

\subsubsection{Intercambio de iteraciones (loop exchange)}

\subsubsection{Unión de bucles (loop fusion)}

\subsubsection{Blocking}

\subsubsection{Búsqueda anticipada}


\section{Reducción de la Penalización por Fallo de Caché}

\section{Reducción del Tiempo en Caso de Acierto en Caché}

\section{Organizaciones de la Memoria Principal}

\bibliographystyle{plain} 
\bibliography{bibliografia} 

\end{document} 
